\chapter{Methods}

This study implements three research methods to gain a broader understanding of the possibilities, state of research and technological practices regarding Progressive Web Apps.

\section{Literature Search}

A search for "Progressive Web Apps" in Google Scholar returns

Notably, there is a book by liebl in German \cite{liebelProgressiveWebApps2019}

Interestingly, there seems to be no study available yet as
Further reserach will have to


\section{Related Works}
The Google Web Fundamentals group acts as the driving force behind the documentation on PWA related topics and the creation of blog posts and tutorials. As the supporting technologies advance with every Chrome browser release, the Web Fundamentals website\footnote{\url{https://developers.google.com/web/fundamentals}} is the foundation for upcoming research and studies.

Other than Google-created content, online platforms such as Medium blog posts\footnote{\url{https://medium.com/search?q=Progressive\%20Web\%20Apps}} and Youtube tutorials\footnote{\url{https://www.youtube.com/results?search_query=Progressive+Web+Apps}} lay the foundation of technical research for this study. Notably, the technical series of articles about PWA published by Eder Ramírez Hernández\footnote{\url{https://medium.com/@eder.ramirez87}} on Medium were extremely informative.

\section{Technical Implementation}
To gain a better understanding of the possibilities of Progressive Web Apps a PWA was developed. The motivation for the PWA was to take an existing native application as a reference and to re-create the application as a Progressive Web App. For this purpose I picked the \textit{Beer With Me}\footnote{\url{https://beerwithme.se/}} application, which is available on the Apple App Store as well as on Google Play. As stated by the Swedish developers Antonsson, Knutsson and Tidbeck "Beer With Me is a social application to notify your friends when you are drinking so that they can join".
