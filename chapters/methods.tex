\chapter{Methods}

This study implements three research methods to gain a broader understanding of the possibilities, state of research and technological practices regarding Progressive Web Apps.

\section{Literature Search}

Our searches for academic involvement in Progressive Web Apps returned a very limited amount of results as per July 2017 \citep{biorn-hansenProgressiveWebApps2018}. The same search performed 2 years later doesn't reveal any further major releases on this topic. Scientific research is still pretty sparse. However, technology related publishers have picked up the topic with a handful of publications in 2017 and 2018 \citep{humeProgressiveWebApps2018,aterBuildingProgressiveWeb2017}. In 2019 though, there seems to be only one notable publication; a practical guide to PWAs by Liebel written in the German language \citep{liebelProgressiveWebApps2019}.

Interestingly, I was not able to find any scientific research on the commercial aspects of using PWA. A comparison on budgets and running costs required by native/hybrid apps versus a PWA. As noted in the conclusion of this study this leaves room for future research.

\section{Related Works}
The Google Web Fundamentals group acts as the driving force behind the documentation on PWA related topics and the creation of blog posts and tutorials. As the supporting technologies advance with every Chrome browser release, the Web Fundamentals website\footnote{\url{https://developers.google.com/web/fundamentals}} is the foundation for upcoming research and studies.

Other than Google-created content, online platforms such as Medium blog posts\footnote{\url{https://medium.com/search?q=Progressive\%20Web\%20Apps}} and Youtube tutorials\footnote{\url{https://www.youtube.com/results?search_query=Progressive+Web+Apps}} lay the foundation of technical research for this study. Notably, the technical series of articles about PWA published by Eder Ramírez Hernández\footnote{\url{https://medium.com/@eder.ramirez87}} on Medium were extremely informative.

\section{Technical Implementation}
To gain a better understanding of the possibilities of Progressive Web Apps a PWA was developed. The motivation for the PWA was to take an existing native application as a reference and to re-create the application as a Progressive Web App. For this purpose I picked the \textit{Beer With Me}\footnote{\url{https://beerwithme.se/}} application, which is available on the Apple App Store as well as on Google Play. As stated by the Swedish developers Antonsson, Knutsson and Tidbeck "Beer With Me is a social application to notify your friends when you are drinking so that they can join".
