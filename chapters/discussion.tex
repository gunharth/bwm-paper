\chapter{Discussion}

In conclusion, PWAs might not replace native apps completely within the next few years. However, the techniques involved have leveled up the standard of modern websites. Rapid enhancements of browser technologies and related standards promise further integration with browsers and devices.

The new Twitter website demonstrates that PWAs stepped out of the teething phase and are a serious contender for platform independent production ready applications. It is expected that this will entice other businesses to implement PWA technologies in the near future in order to streamline the delivery of their products.

A tighter integration for PWA development into existing frameworks, plugins and service providers will be necessary to raise the awareness for PWAs as a solid alternative to native and hybrid approaches.

Next too technical research on PWAs there is plenty of room for related future studies. The new technology introduces new user behavior in the likes of installing an app on a desktop PC, which implies studies on user behavior and usability. Further, PWAs have the potential to make app stores obsolete. The financial implications need to be researched and studies on possible future commercializations of PWAs are recommended.
