\chapter{Introduction}

In 2015 the Google Developer Alex Russel first drafted the term Progressive Web Apps (PWAs) and formulated the concept of PWAs as it stands today.

Full Progressive App support will distinguish engaging, immersive experiences on the web from the “legacy web”. Progressive App design offers us a way to build better experiences across devices and contexts within a single codebase but it’s going to require a deep shift in our understanding and tools. \citep{russellProgressiveWebApps2015}

Since then, notably Google and its Chrome browser focus on the implementation of all necessary technologies in order to fully support PWAs and it's technologies. Nearly every new release of the Chrome browser adds further support for PWAs. As an example, at the time of writing Chrome 76 shows an install icon, if a site meets the Progressive Web App installability criteria\footnote{\url{https://developers.google.com/web/fundamentals/app-install-banners\#criteria}}.

So far, engaging and immersive experiences has been occupied by the development of mobile apps. Native as well as hybrid technologies are used to satisfy the requirements in this field. However, developing a native application is an expensive solution. Programmers spend years learning to code heavy native apps and app owners usually invest a huge budget finding a development team \citep{nguyen2019progressive}. That said, at least two version for iOS and Android need to be developed, maintained and published to the App Stores accordingly.

As PWAs are still a rather new concept there are known limitations in terms of unified browser support of APIs\footnote{\url{https://whatwebcando.today}}, missing filesystem access and platform-level features like calendar or contacts \citep{biorn-hansenProgressiveWebApps2018,malavoltaNativeAppsWeb2016}. However, there are many applications that do not depend on the before mentioned limitations. On 15th July 2019 Twitter announced it's new website, which is a PWA build on one code base feeding all devices and browsers that are accessing the site on the Web. The goal for the new site was to make it easier and faster to develop new features for people worldwide and to provide each person and each device with the right experience \citep{croomBuildingNewTwitter2019}.

The purpose for this study is to look at the technical side of PWAs, whereby the following questions will be answered:
Is it possible to take an existing mobile app as a prototype and re-build the app as a PWA?
What services and programming languages are best fitted in doing so?
How do those services and programming languages support the development of a PWA?
What are the technical limitations, if any?
