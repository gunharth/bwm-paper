\chapter{Introduction}

\blindtext \citet{Shearer2000}

\section{Problem Situation}

\blindtext

\fig{img/sax_approximated_series}{Sax approximation of a time series}{fig:sax}{0.5}

As can be seen in Figure~\ref{fig:sax} \ldots

\section{Objectives}

\blindtext

\section{Methods}

\blindtext

\section{Structure}

\blindtext

\section{Tables}

Table~\ref{tab:table-one} shows an example table.

\begin{table}[htbp]
    \centering
    \caption{This is a table}
    \label{tab:table-one}
    \begin{tabular}{lll}
        \addlinespace
        \toprule
        Column 1 & Column 2 & Column 3 \\
        \midrule
        A     & B     & C \\
        D     & E     & F \\
        G     & H     & I \\
        \bottomrule
    \end{tabular}
\end{table}

\section{Source Code}

\begin{lstlisting}[language=Java, caption=Hello World in Java, label=lst:hello-world-java]
public class Hello {
    public static void main(String[] args) {
        System.out.println("Hello World");
    }
}
\end{lstlisting}

Listing~\ref{lst:hello-world-java} shows the classic Hello World in Java.

\lstinputlisting[language=Python, caption=Hello World in JavaScript, label=lst:hello-world-py]{./lst/hello.py}

Listing~\ref{lst:hello-world-py} shows the classic Hello World in Python.
