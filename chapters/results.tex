\chapter{Results}

Overall, re-building the prototype of the original application was a success. The foundational question of this study was if it will be possible to duplicate the features of the original application as a PWA. This can be answered in a positive way.

Frontend frameworks, like the one used in this study, strive to support the development of optimized PWA skeletons to streamline the development and promise future enhancements to reducing time to market (TTM)\footnote{\url{https://en.wikipedia.org/wiki/Time_to_market}}. It is assumed that this tendency continues to unveil further productive environments.

Firebase, as a representative of a solution provider for server-less application delivery is an example of a rather new industry. It is expected that Firebase will expand on its services and other service providers will enter this market in the near future to cater for the quickly rising demand in this field.

In terms of the actual development process, more streamlined processes are required. The modularity of npm packages still seems to be to scattered and present a hurdle for rapid developments. Offering better integrated, modular extensions should be the focus of frontend frameworks as well as cloud specific service providers. Overall, it is observed that in some cases reading the documentation is a time consuming task and a distraction from the actual development of a product.

Finally, it is advised to closely look at the application requirements at an early stage. Not every project is suitable to be developed as a PWA yet. Hence, if an application depends on platform or device specific services native and hybrid solutions still are a better choice.
