
% This thesis template is based on the report class of
% latex. Yout do not need to change this.
\documentclass[12pt, a4paper]{report}

% This template uses the babel package for customizing
% labels like Figure/Abbildung etc. Please adust the
% language parameter if necessary. Currently, it is
% set to 'english'. If you write in German adjust the
% parameter to 'ngerman'
\usepackage[english]{babel}

% This includes all variables that are used in this work
% Use this document to specify your name, your reviewers
% and other values that you would like to maintain
% in one place.
\newcommand\myname{Gunharth Randolf}
\newcommand\mypkz{1810738273}

\newcommand\myfirstreviewer{Prof. (FH) Dr. Michael Kohlegger}
\newcommand\mysecondreviewer{Lukas Demetz, PhD}

\newcommand\mytitle{"Hello Desktop PWAs" - A Proof of Concept}

% The value below is the date that you finished your work. This
% date appears on you work's cover page and in the "Eidesstattliche
% Erklärung".
\newcommand\mydate{29. July 2019}

% This template assumes that you are studying Data Science &
% Intelligent Analytics. Please Change if necessary
\newcommand\academictitle{}
\newcommand\worktype{}
\newcommand\studyprogram{Web Communication \& Information Systems}


% This includes all packages and presets. You do not need
% to change this file unless you want to add new packages
% or change the presets of one of the used packages.
%\usepackage{bookman}  % Font package
%\usepackage[light]{CrimsonPro}
\usepackage{pxfonts}
\usepackage[T1]{fontenc}
\usepackage{graphicx}
\usepackage[utf8]{inputenc}
\usepackage{blindtext}  % Used for dummy text segments
\usepackage{natbib}
\usepackage{hyperref}
\usepackage{booktabs}
\usepackage{xcolor}
\usepackage{listings}
\lstset{numberbychapter=false}
\iflanguage{english}{\renewcommand{\lstlistlistingname}{List of Listings}}{\renewcommand{\lstlistlistingname}{Listingsverzeichnis}}

\usepackage{fancyhdr}  % Head and footer styling
\fancyhead[RO]{\thepage}
\fancyhead[LO]{\nouppercase\leftmark}
\renewcommand\headrulewidth{0.5pt}
\renewcommand\footrulewidth{0pt}

\usepackage[a4paper, left=3.5cm, right=3cm, top=3.5cm, bottom=3cm]{geometry}
\setlength{\headheight}{15pt}

\usepackage[title, titletoc, header]{appendix}
\renewcommand{\appendixname}{Appendix}

\usepackage{titlesec}
\titleformat{\chapter}{\bf \LARGE}{\thechapter.}{16pt}{\LARGE}

\usepackage{chngcntr}
\counterwithout{figure}{chapter}
\counterwithout{table}{chapter}

\bibliographystyle{apalike}
\pagestyle{headings}
\setlength{\parindent}{0em}
\setlength{\parskip}{1.5em}
\renewcommand{\baselinestretch}{1.13}

\newcommand\frontmatter{
    \pagestyle{fancy}
    \fancyfoot{}
    \pagenumbering{Roman}
}

\newcommand\mainmatter{
    \pagenumbering{arabic}
}

\newcommand\backmatter{
    \pagenumbering{arabic}
    \renewcommand{\thepage}{A\arabic{page}}
}

\lstset{
    numberstyle=\tiny,
    numbers=left,
    showstringspaces=false,
    breaklines=true,
    commentstyle=\itshape\color{darkgray},
    basicstyle=\ttfamily,
    stringstyle=\color{orange},
    keywordstyle=\bf\color{green!40!black},
    identifierstyle=\color{blue}
}

\hypersetup{
  pdftitle    = \mytitle,
  pdfsubject  = \worktype,
  pdfauthor   = \myname,
  pdfcreator  = {pdflatex using template provided by University of Applied Sciences FH Kufstein Tirol},
  bookmarksnumbered = true,
  colorlinks = true,
  linkcolor = blue,
  citecolor = green,
  urlcolor = orange
}


% This includes custom latex commands. You can use this
% file to create your own command sequences.
\include{commands}

% If you want to print your thesis in black and white
% uncomment the following section

% \hypersetup{
%     colorlinks=false,
%     pdfborder = 0 0 0
% }
% \lstset{
%     commentstyle=\itshape,
%     basicstyle=\ttfamily,
%     stringstyle=\color{black},
%     keywordstyle=\bf,
%     identifierstyle=\color{black}
% }

\begin{document}

    \frontmatter

    % This places the front matter. You do not need to
    % change this file. All thesis-specific values are
    % imported from ``core/variables.tex``.
    \begin{titlepage}

    \vfill
    \begin{center}
      \includegraphics[width=4.5cm]{img/kufstein_logo.png} \\
    \end{center}
    \vfill

    \begin{center}
      \Large \textbf{\mytitle}
    \end{center}
    \vfill

    % \begin{center}
    %   %\condMASTER{\Large Masterarbeit}{\Large Bachelorarbeit}
    %   \worktype
    % \end{center}
    % \vfill

    % \begin{center}
    %  zur Erlangung des akademischen Grades\\
    %  \large \textbf{\academictitle}
    % \end{center}
    % \vfill

    \begin{center}
      % As part of the lecture:\\
      \textbf{\myname}\\
      \vspace{0.1cm}
      \textbf{\mypkz}\\
      \vspace{0.1cm}
      \textbf{Entwicklung \& Betrieb Mobiler Informationssysteme}\\
      \vspace{0.1cm}
      \textbf{\studyprogram}\\
      \vspace{0.1cm}
      \textbf{\mydate}
    \end{center}
    \vfill
  \end{titlepage}


    % This places the "Eidesstattliche Erklärung". This
    % Document is always in German, even if your work is
    % written in English.
    % \chapter*{Eidesstattliche Erklärung}
\thispagestyle{empty}

Ich erkläre hiermit, dass ich die vorliegende Masterarbeit selbständig und ohne fremde Hilfsmittel verfasst und in der Bearbeitung und Abfassung keine anderen als die angegebenen Quellen oder Hilfsmittel benutzt sowie wörtliche und sinngemäße Zitate als solche gekennzeichnet habe. Die vorliegende Masterarbeit wurde nicht anderweitig für Prüfungszwecke vorgelegt.

\vspace{2cm}
Kufstein, \mydate

\vspace{2cm}
\rule{10cm}{1pt}\\
\myname{}



    % This will include a "Sperrvermerk". This document
    % is always in German, even if your work is written
    % in English. Please remove this line, if necessary.
    % \include{frontmatter/sperrvermerk}

        % This adds a german and an english summary to your work.
        % You always have to add both, independent of whether your
        % work is done in german or english
        % \thispagestyle{empty}

\textbf{FH Kufstein Tirol\\Data Science \& Intelligent Analytics}

Abstract of the thesis: \textbf{\mytitle}

\textbf{Author:} \myname\\
\textbf{First reviewer:} \myfirstreviewer\\
\textbf{Second reviewer:} \mysecondreviewer\\

% Your text goes here (aprox. 350 words)
\Blindtext[2][1]

% End with date
\mydate{}

        % \thispagestyle{empty}

\textbf{FH Kufstein Tirol\\Data Science \& Intelligent Analytics}

{Kurzfassung der Masterarbeit: \textbf{\mytitle}}

\textbf{Verfasser:} \myname\\
\textbf{Erstgutachter:} \myfirstreviewer\\
\textbf{Zweitgutachter:} \mysecondreviewer\\

% Your text goes here (aprox. 350 words)
\Blindtext[2][1]

% End with date
\mydate{}


        \thispagestyle{empty}

\textbf{Abstract}

Lorem ipsum


    % This inserts all necessary tables into your work.
    \tableofcontents
    % \listoffigures
    % \listoftables

    % Use this if you have any code listings
    % \lstlistoflistings

    \mainmatter

    % This places the actual chapters. The files referenced here
    % are just an example. You can add additional chapters if
    % necessary
    % \include{chapters/chap01}
    % \include{chapters/chap02}
    % \include{chapters/chap03}
    % \include{chapters/chap04}
    % \include{chapters/chap05}
    \chapter{Introduction}


In 2007 announces Steve Jobs iphone ...

Latest advances in web technologies due to the support by Google pushing the support.

Recent applications like the one from Twitter

What are progressive web apps.

Motivation
\cite{liebelProgressiveWebApps2019}

The ACM Digital Library only shows 6 search results since 2017
A search on Google scholar revealed All articles related to technology side of things. Surprisingly, I was not able to find any entries that focussed on the business side of things. Just looking at the costs of developing and maintaining native or hybrid apps

Developing a native application is an expensive solution. Programmers spend years learning to code heavy native apps and app owners usually invest a huge budget finding a development team. Therefore, an answer to this persistent problem is enhancing traditional web app to a Progressive Web App (PWA). \citep{nguyen2019progressive}

Is it possible to take an existing mobile app and convert it to a PWA?
What services and programming languages are a best fit in doing so?
How do those services and programming languages support the creation of a PWA?
What are the technical limitations, if any?



For this study a PWA is created and deployed on the Internet.

The purpose for this study is to answer the following questions:

Is it possible to take an existing mobile app as a prototype and re-build the app as a PWA?
What services and programming languages are a best fit in doing so?
How do those services and programming languages support the development of a PWA?
What are the technical limitations, if any?


    % This places the bibliography. You can add more
    % bibliographic items it the bibliography.bib file.
    % We suggest using a reference manager (e.g. Jabref)
    % to maintain this file
    \bibliography{bibliography}

    \newpage

    \begin{appendices}

        \backmatter

        % This is the actual appendix. The files referenced here
        % are just examples. You can add additional appendices
        % if necessary
        % \chapter{List of Interview Parners}

\Blindtext[2][6]

        % \include{appendix/app02}

    \end{appendices}

\end{document}
